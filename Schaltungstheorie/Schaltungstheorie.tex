\documentclass{article}
\usepackage{amsmath}
\usepackage{amssymb}
\title{Zusammenfassung der Wissenstruktur Schaltungstheorie}
\begin{document}
    \maketitle
    \pagebreak
    \tableofcontents
    \pagebreak
    \section{Tor}
    \textbf{Tor-Theorie}: abstrakte Betrachtung eines Bauelements mit $u$ und $i$ nach festgelegter Richtung (sog. Zählpfeil)
    \subsection{Geltungsbereich der Tor-Theorie}
    \subsubsection*{Konzentriertheitshypothese}
    Beschränkung der Bauelements Ausdehnung: $d \gg \lambda; \lambda = \frac{c}{f}\rightarrow c\ll d\cdot f$
    \subsubsection*{Torbedingung erfüllte Bauelemente}
    Für einen Eintor(Zweipol) gilt: $i_{ein} = i_{aus}$
    \subsection{Allgemeine Gesetze eines Tores}

    
\end{document}